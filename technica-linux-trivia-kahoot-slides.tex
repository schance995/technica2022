% Created 2022-10-08 Sat 19:53
% Intended LaTeX compiler: pdflatex
\documentclass[presentation]{beamer}
\usepackage[utf8]{inputenc}
\usepackage[T1]{fontenc}
\usepackage{graphicx}
\usepackage{longtable}
\usepackage{wrapfig}
\usepackage{rotating}
\usepackage[normalem]{ulem}
\usepackage{amsmath}
\usepackage{amssymb}
\usepackage{capt-of}
\usepackage{hyperref}
\hypersetup{colorlinks=true}
\usetheme{default}
\date{\today}
\title{Linux Trivia Kahoot}
\hypersetup{
 pdfauthor={},
 pdftitle={Linux Trivia Kahoot},
 pdfkeywords={},
 pdfsubject={},
 pdfcreator={Emacs 29.0.50 (Org mode 9.5.5)}, 
 pdflang={English}}
\begin{document}

\maketitle

\begin{frame}[label={sec:orgd297123}]{Who uses Linux?}
Lots of people!

\begin{itemize}
\item Every big tech company uses Linux in production and contributes to it: \textasciitilde{}75\% of Linux code is written by corporation employees!
\item Linux is installed on zillions of devices: routers, embedded systems, servers, etc.
\item Hundreds of thousands of people use Linux on their personal devices (extrapolated from the \textasciitilde{}1\% of Steam users using Linux).
\end{itemize}

\href{https://findly.in/how-many-linux-users-are-there/}{Further reading}.
\end{frame}

\begin{frame}[label={sec:org828d538}]{Notable mention: BSD}
The Berkeley Software Distribution is another Unix-like operating system family. Most Linux software has been ported to BSD.

\begin{enumerate}
\item FreeBSD = general purpose
\item OpenBSD = security
\item NetBSD = portability
\end{enumerate}
\end{frame}

\begin{frame}[label={sec:org4e04856}]{What is Free Software?}
Software that meets 4 essential freedoms:

\begin{enumerate}
\setcounter{enumi}{-1}
\item The freedom to run the program for any purpose.
\item The freedom to study how the program works, and change it as you wish.
\item The freedom to redistribute copies.
\item The freedom to distribute copies of your modified versions.
\end{enumerate}

(Adapted from \href{https://www.gnu.org/philosophy/free-sw.en.html}{What is free software?})
\end{frame}


\begin{frame}[label={sec:org52739b9}]{Examples of libre software that run on Linux}
\begin{itemize}
\item VLC media player
\item All major browser engines:
\begin{itemize}
\item Google Chromium, the libre software base for nonfree Google Chrome
\item Mozilla Firefox
\item WebKit, the libre software base for nonfree Safari
\end{itemize}
\item Almost all programming languages
\begin{itemize}
\item Python, R, Rust, Go, C, C++, Java, JavaScript, Ruby, etc
\end{itemize}
\item Almost all command line tools
\begin{itemize}
\item Git, OpenSSH, Apache, Nginx, Node, etc
\end{itemize}
\end{itemize}
\end{frame}

\begin{frame}[label={sec:org3f6cafb}]{Major licenses}
\begin{itemize}
\item proprietary: not allowed to make changes
\item GPL: changes you make must be shared
\item BSD: changes you make do not have to be shared
\item Public domain: no copyright owner
\end{itemize}
\end{frame}

\begin{frame}[label={sec:org7b18c15}]{The shell is the primary interactive program in the terminal}
\begin{enumerate}
\item Bash: the GNU shell.
\item Zsh: very customizable.
\item Tcsh: uses C-style syntax.
\item Fish: lots of goodies out of the box, such as syntax highlighting and autosuggestions.
\end{enumerate}
\end{frame}

\begin{frame}[label={sec:org788a7f4}]{Commands}
\begin{itemize}
\item Navigate terminal
\item Copy and delete files
\item Installing software
\end{itemize}
\end{frame}

\begin{frame}[label={sec:org1454ab8}]{Distributions}
Since Linux is just a kernel, it is distributed with useful software. Thus the collection of the kernel and software is called a distribution (aka distro).

\begin{itemize}
\item \emph{Independent} distros started by compiling the kernel and programming language toolchains from scratch. \emph{Derivative} distributions use independent distros as a base for further customization. For example, Debian is independent and Ubuntu is a derivative of Debian.
\item \emph{Binary-based} distros distribute pre-compiled software. \emph{Source-based} distros expect the user to compile it from source, which takes longer but is more customizable. For example, Arch is binary-based and Gentoo is source based.
\end{itemize}
\end{frame}

\begin{frame}[label={sec:org0f4b767}]{Desktop}
A small subset of Linux desktops:

\begin{center}
\begin{tabular}{llll}
Desktop & UI toolkit & Install Size & Supports Wayland?\\
\hline
GNOME & GTK & Large & Yes\\
KDE & QT & Large & Yes\\
XFCE & GTK & Small & No\\
LXQT & QT & Small & No\\
\end{tabular}
\end{center}

\begin{itemize}
\item You can always install a new desktop and select it from the login menu.
\item Larger desktops tend to come with more features preinstalled.
\item Wayland and Xorg are the two major display architectures for Linux graphics. It's important to know which one you're using when troubleshooting.
\end{itemize}
\end{frame}

\begin{frame}[label={sec:orgd2c1c13}]{Fun facts}
\begin{enumerate}
\item Linux Torvalds is the lead developer and maintainer of Linux.
\item The Linux mascot is a penguin.
\end{enumerate}
\end{frame}

\begin{frame}[label={sec:org5d92043}]{What next?}
Number 1 tip as a developer: know how to get help!

\begin{enumerate}
\item Read the documentation
\item Search the internet
\item Ask for help on forums, by email, or by internet chat
\end{enumerate}

Good luck on starting your Linux journey! If you are a UMD student you can join the \href{https://terplink.umd.edu/organization/linux-club-at-umd}{Linux Club at UMD}.
\end{frame}
\end{document}